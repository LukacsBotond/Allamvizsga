\documentclass[12pt, twosides]{report}

\include{preamble}
\usepackage{hyperref}

\title{
	{Elektronikai alkatrész teszter}\\
	{\large Sapientia\\
	Erdélyi Magyar Tudományegyetem, Marosvásárhely}
}
\author{
	Lukács, Botond\\
	\texttt{lukacs.botond@student.ms.sapientia.ro}
	\and
	dr. Túrós, László-Zsolt\\
	\texttt{turosl@ms.sapientia.ro }	
}
\date{2022}

%%%%%%%%%%%%%%%%%%%%%%%%%%%%%%%%%%%%%%%%%%%%%%%%%%%%%%%%%%%%%%%%%%%%%%%%%
\begin{document}

\includepdf[pages={1,2,3}]{pdfs/allamvizsga_boritoMod.pdf}

\section*{Extras}
În zilile noastre, systeme de microcontrolere pot fi găsite în multe aplicațile
și avea o mare capacitate de calcul. În general pot fi utiliyate într-o
mare verietate de aplicații. Au o mulțime de funcții, de la simpla comutare 
a LED-urilor la automatizarea sistemelor complexe.
În plus pot utiliza accesorii externe cu care pot fi folosite în
mai multe aplicațile.

Ca urmare, scopul disertației este de a dezvolta un sistem care să poată 
determina componente electronice simple și valoarea lor aproximativă, 
precum și alocarea piciorului acestora.
Există multe componente simple în electronică, cum ar fi rezistoare, 
condensatoare, diode, tranzistoare.
Cu toate acestea, atunci când construiți un circuit, este bine să știți 
exact care este acea componentă, acest lucru este valabil mai ales pentru 
diferiți semiconductori.
În multe cazuri, ID-ul componente nu este visibil or fica techica
nu pot fi găsite.
Acesta este scopul „testerului de componente electronice” care determină și 
imprimă automat valorile componentei sau că componenta testată este defectă/nu 
este recunoscută.
Sistemul folosește un microcontroler și un ecran pentru a identifica 
componenta și a afișa datele acesteia către utilizator.
Identificarea este complet automată, tot ce trebuie să faceți este să 
conectați o componentă necunoscută și să comutați un comutator sau să 
porniți automat când este conectat la curent.

În disertației se va ocupa de aplicațiile microcontrolerelor și proiectarea acestora, 
recunoașterea și măsurarea componentelor.

\textbf{Cuvinte cheie}: microcontrolere, tranzistoare, Identificare

\pagebreak

\includepdf[pages={6}]{pdfs/allamvizsga_boritoMod.pdf}

\section*{Kivonat}
Napjainkban a mikrovezérlős rendszereken sok mindenben megtalálhatóak és manapság nagy számítási kapacitással rendelkeznek, általánosan alkalmazhatóak sokféle különböző alkalmazásban. Rengeteg funkciójuk van, az egyszerű LED villogtatástól kezdve komplex rendszerek automatizálásáig. Ezen kívül könnyű külső kiegészítő tartozékokat amelyekkel sokkal szélesebb körben használhatóak.

Ennek hatására a dolgozat célkitűzése egy olyan rendszer kialakítása, amely képes meghatározni egyszerű elektronikai komponenseket és azok megközelítő értékét és ezen kívül a lábkiosztását is amennyiben ez szükséges. Az elektronikában sok féle egyszerű komponenssel találkozhatunk, mint ellenállások, kondenzátorok, tranzisztorok. Viszont egy áramkör építésénél jó tudni, hogy az a komponens pontosan mi, ez legfőképpen igaz a különböző félvezetőkre. Sok esetben az azonosítója lekopott, vagy nem található adatlap így nehéz beazonosítani, hogy pontosan mi az a komponens. Erre szolgál az „elektronikai alkatrész teszter” amely automatikusan meghatározza, vagy kiírja, hogy hibás alkatrész ha nem ismeri fel vagy sérült az tesztelt alkatrész. A rendszer egy mikrovezérlőt és egy kijelzőt használ az komponens azonosítására és arról levő adatok kijelzésére a felhasználó felé. Az azonosítás teljesen automata, csupán csatlakoztatni kell az ismeretlen komponenst és egy gombot megnyomni.

A dolgozatban a mikrovezérlős alkalmazásokról és azok tervezéséről, alkatrészek felismeréséről és méréséről lesz szó.

\textbf{Kulcsszavak}: mikorvezérlő
\pagebreak

\section*{Abstract}
In today's world microcontroller system can be found in a lot of things and
they have a big processing capacity and can be used generally in a lot of
different applications. They have a lot of functions, from making a LED
blink to automating complex systems. They also can be easily connected to
external components so this way they can be used in more ways.

For this cause this dissertation main goal it making a system which will be
able to identify simple electoronic components whith their aproximative
values and their pinouts.
In electoronics there are a lot of electronic components, like resistors,
capacitors, diodes and transistors. But when building a circuit
it is good to know what is the component and their pinout, which is more important
on semiconductors. In a lot of times the identifier values are unreadable
or there is no datasheet avabile so it is hard to idnetify the component.
For this purpose the "electronic component teszter" was made which automaticall
identif and outputs the component values or outputs if if failed to identify
the component. The process if fully automatic, only the component need to be 
connected to the tester socket and togle switch or it automatically starts
when it is connected to power.

The dissertation will cover microcontroller applications and their design
for component recognition and measurement.


\textbf{Keywords}: microcontroller, transistor, identifying
\pagebreak

\pagenumbering{gobble}

\tableofcontents

\listoffigures

\chapter{Bevezető}
\pagenumbering{arabic}

A mikrovezérlős rendszerek manapság az életünk minden részében megtalálhatóak, kis méretük, 
alacsony áruk és meglehetősen nagy teljesítményükkel sok mindenre általánosan használhatóak. 
Ez nagyban csökkenti a tervezési költségeket, mivel nem kell egy specifikus logikai áramkört 
kialakítani minden egyes alkalmazási területre, csupán új program kódot kell feltölteni és 
használható egy teljesen más célra.

A mikrovezérlő könnyen összeköthetőek külső eszközökkel amellyel rengeteg mindent meg lehet 
valósítani és bővíteni a lehetőség és szabadon vezérelhetők a GPIO-n (General Purpose Input 
Output) keresztül sok mindent el lehet érni, az egyszerű LED kapcsolgatásától komplex jelek 
generálásáig. Általános esetben a mikrovezérlők csak a legfontosabb részeket tartalmazzák, 
mint az Analog Digital Converter amivel egy analóg jelet alakít egy digitális jellé amit a 
processzor fel tud majd dolgozni, ez legfőképpen azért van, mert nem mindenkinek van szüksége 
mindenre így akinek szüksége van az egyszerűen az külsőleg csatolja hozzá. 

Az összeállított rendszert szabadon lehet vezérelni így automatizálni lehet vele folyamatokat 
amivel egyszerűsíti az emberek dolgát. Viszont nagyban gyorsítja a folyamatok sebességét és 
pontosságát, miközben csökkenti a költségeket, mivel nem kell egy képzett dolgozó irányítsa, 
egy ilyen példa a robotkarok irányítása, lehetséges lenne karok segítségével irányítani, 
viszont ez lassú és költésges, mivel minden robotkar mellé kellene egy munkás aki közel sem 
lenne elég nagy pontosságú. 

Ezen kívűl gyakran használják automatizált tesztelésekre is, mivel egyszerű megismételhető 
teszteket végezni velük, miközben képesek valós időben mérni a rendszer viselkedését. Ennek 
a feladatnak is ez a lényege, az alkatrészek tesztelése egy gyors és automatizált módon. 


\section{Téma meghatározása}

A dolgozat célja egy olyan eszköz tervezése, amit bárki elektronikai ismeret nélkül is egyszerűen 
használni lehet. Sok esetben a feliratok az alkatrészeken nehezen látható, lekopott vagy 
egyszerűen nincs feltüntetve. Ilyen esetben sok segítséget tud nyújtani egy olyan eszköz ami 
gyorsan meg tudja határozni a komponenst és a lábkiosztását is amennyiben ez fontos.

Ez különösen nagy segítséget nyújt kezdőknek akik még kevésbé ismerik az alkatrészeket és az 
adatlapjai meg nagyok és komplexek számukra és a fő információk megjelenítése néhány sorban. 
Haladóknak is nagy segítség, mivel az ellenállást színkódjátról egyszerű meghatározni, viszont 
a teszterrel meg lehet határozni, hogy az alkatrész hibás-e, vagyis ha a tranzisztor kiégett 
akkor az is letesztelhető.

A teszternek 3 teszt terminálja van, ebbe kell az ismeretlen komponenst bekötni és képes 
egyszerű elektronikai komponensek (ellenállás, dióda, tranzisztorok, stb.) automatikus 
felismerését és az adatainak meghatározására. Viszont nem képes bonyolultabb áramkörök 
azonosítására aminek összesen tübb mint 3 lába van.

Megvalósítás során a költégek csökkentése a cél, miközben a pontosság nem csökken nagyban. 
Két verzió is összeállítható, az egyik egyszerű ellenállásokkal és a második egy DAC (Digital 
Analog Converter) segítségével. Mindkettő alkalmas a komponensek meghatározására, viszont az 
ellenállásos verzió nem alkalmas karakterisztika diagramm kirajzolására, viszon sokkal olcsóbb, 
mivel nem használ egy külső DAC-ot.

Mérés eredménye kikerül egy kis kijelzőre és grafikus felületen is megtekinthető amennyiben 
egy számítógéphez van csatolva. Viszont a 2 közül legalább az egyikre szükség van, különben a 
mérés eredménye nem lesz látható. A kijelzőt nem kötelező alkalmazni, viszont annélkül csak 
egy laptop/számítógéphez kapcsolva lehet használni.

Ehhez szükséges egy processzor, viszont manapság a mikrovezérlők nagy számítási kapacitással 
rendelkeznek meglehetősen alacsony áron és néhány ellemállásból és vezetékből otthon is 
összeállítható.

Karakterisztika diagramm kirajzolása is fontos, viszont ez leginkább a tranzisztoroknál 
fontos, mivel az ellenállások lineárins összefüggést mutatnak a feszültség és áramerősség közt 
és a diódák meg magas áram növekedést ami után a feszültésg elérte a nyitó feszültséget.

A rendszer táplálása bármilyen USB csatlakozón keresztül lehetséges, mivel ez széles körben 
megtatlálható vagy külső akkumulátor is megfelel amelynek van USB kimenete.

\section{Hasonló eszközök}

Just \hyperlink{label1}{Click me!}. In the paper \cite{Test00}...



Ez a verzió egy egyszerűsített verzió, viszont ez is képes felismerni a komponenst és meghatározni a paramétereit, viszont kissebb a pontossága és nem képes karakterisztika diagrammot kirajzolására.

A bekötés a következőképpen lehetséges (lásd \ref{fig:basicTesterConnection} ábra).

\begin{figure}
    \centering
    \hfill
    \subfigure[Insbot]{\includegraphics[scale=0.3]{figures/images/literature/BasicTesterConnection.png}}
    \caption{Ellená}
    \label{fig:basicTesterConnection}
\end{figure}



%Két ábra egymás mellett (lásd \ref{fig:insbots} ábra).

%\begin{figure}[h]
%    \centering
%    \hfill
%    \subfigure[Insbot \cite{colot2004insbot}]{\includegraphics[scale=0.3]{figures/images/literature/insbot.png}}
%    \hfill
%    \subfigure[Insbot és csótányok interakciója \cite{garnier2011ants}.]{\includegraphics[scale=0.25]{figures/images/literature/insbot_cockroach.png}}
%    
%    \caption{Insbot és csótányok interakciója. Az insbot-ok képesek a csótányokat csalogatni.}
%    \label{fig:insbots}
%\end{figure}

\chapter{Szakirodalom áttekintése}
\section{Cím 1}

\subsection{Alcím 1}


\chapter{Elméleti áttekintés}
\input{chapters/theory.tex}

\chapter{Rendszer specifikációi}
\input{chapters/specifications}

%Terv
\chapter{Gyakorlati megvalósítás} \label{chpt:implementation}
\input{chapters/implementation}

\chapter{Eredmények}
\section{Cím 1}

Eredmények leírása



%következtetés
\chapter{Összefoglalás}
\input{chapters/summary}

%irodalomjegyzek
\addcontentsline{toc}{chapter}{Irodalomjegyzék}
\begin{thebibliography}{99}
\bibitem{Test00} T. Test, \emph{Test testing tests}, Journal of Testing
%\bibliographystyle{ieeetr}
%\bibliography{References}
\end{thebibliography}

%appendix
\appendix
\chapter{Függelék}
\input{chapters/appendix}



\end{document}
