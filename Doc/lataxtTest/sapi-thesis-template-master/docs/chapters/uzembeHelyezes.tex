A teszter elkészítése után a rendszer tesztelése következett.
A rendszer nem fektetett nagy hangsúlyt a pontossárga, viszont egy 
megközelítő értéket meg kell tudjon határozni.

\subsection{Ellenállás mérés}

Amennyiben az elkatrész egy ellenállás akkor meghatározza ennek az ellenállását.


\begin{table}[]
    \begin{tabular}{|l|l|l|l|}
    \hline
    Sorszám & Névleges érték & Multiméterrel mért érték & Teszter által mért érték \\ \hline
    0       & 47             & 47                       & 52.5                     \\ \hline
    1       & 56             & 55.7                     & 59                       \\ \hline
    2       & 100            & 100.1                    & 107.4                    \\ \hline
    3       & 220            & 218                      & 242                      \\ \hline
    4       & 360            & 356                      & 398                      \\ \hline
    5       & 540            & 544                      & 606                      \\ \hline
    6       & 1000           & 980                      & 890                      \\ \hline
    7       & 2200           & 2198                     & 2301.7                   \\ \hline
    8       & 4700           & 4670                     & 4319.7                   \\ \hline
    9       & 10000          & 9680                     & 9237                     \\ \hline
    10      & 33000          & 32660                    & 28305.7                  \\ \hline
    11      & 62000          & 62000                    & 49590                    \\ \hline
    \end{tabular}
    \end{table}

\subsection{Kondenzátor}

\begin{table}[]
    \begin{tabular}{llll}
    sorszám & Névleges érték & Mérőműszerrel mért érték & Mért érték \\
    1       & 60nF           & 59.62nF                  & 60nF       \\
    2       & 75nF           & 74.26                    & 75nF       \\
    3       & 100nF          & 98.26nF                  & 101nF      \\
    4       & 200nF          & 202nF                    & 203nF      \\
    5       & 300nF          & 300.3nF                  & 305nF      \\
    6       & 400nF          & 400nF                    & 406nF      \\
    7       & 500nF          & 498nF                    & 504nF      \\
    8       & 600nF          & 602nF                    & 610nF      \\
    9       & 700nF          & 701nF                    & 703nF      \\
    10      & 800nF          & 791.4nF                  & 800nF      \\
    11      & 900nF          & 889.7nF                  & 889nF      \\
    12      & 1uF            & 979nF                    & 1.059uF    \\
    13      & 2uF            & 1.97uF                   & 2.033uF    \\
    14      & 3uF            & 2.95uF                   & 3.05uF     \\
    15      & 4uF            & 3.96uF                   & 4.025uF    \\
    16      & 5uF            & 4.94uF                   & 5uF        \\
    17      & 6uF            & 5.935uF                  & 6.016uF    \\
    18      & 7uF            & 6.914uF                  & 6.95uF     \\
    19      & 8uF            & 7.946uF                  & 8.05uF    
\end{tabular}
\end{table}

\subsection{Karakterisztika diagramm}

