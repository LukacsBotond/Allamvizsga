\section{Megvalósítások}

A projekten belül egy olyan rendszer megvalósítására került sor,
amellyel leegyszerűsíthetőek az alaktrészek azonosítása és mérése.




\section{További fejleszési irányok}

A rendszernek lehetősége volna komplexebb alkatrészek
tesztelésére is, viszont a 3.3V-os feszültség határ
korlátoz sok esetben is. Például a MOSFET-ek tesztelése nagyobb
feszültségre lenne szükség, hogy biztosan azonosíthatóak legyenek.
Ezen kívül több féle dióda és tranzisztor fajták vannak, 
amelyekre komplexebb azonosítási algoritmus szükséges.

Lehetséges lenne frekvencia mérésére is és tekercsek mérésére
is állítható frekvenciával. Rengeteg dolog van ami fejlesztehető
egy viszonlag egyszerű áramkörrel.


