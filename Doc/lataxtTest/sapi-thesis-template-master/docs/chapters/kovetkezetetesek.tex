\section{Megvalósítások}

A projekten belül egy olyan rendszer megvalósítására került sor,
amellyel leegyszerűsíthetőek az alaktrészek azonosítása és mérése.
A megvalósítás során sikerült az egyszerű alkatrészek azonosítása
és mérése, miközben megközelítőleg az értékei is meghatározására
kerültek.

A megvalósítás során fő szempont volt az egyszerűség a felhasználó felé,
így nem kell tudjon semmit az eszközről és elvégezni komplex műveleteket,
hogy az eredményt megkapja. 


\section{Hasolnó rendszerrel való összehasonlítás}

A hasonló rendszer ami alapján terveztem a rendszert \cite{similarSystem}
több dolog detektálására is képes, viszont az egy nyílt forráskódú
projekt, amit többen it tovább fejlesztettek az idők során. Viszont
ebből csak részben a mérő áramkör ami hasonló, pontosabban ellenállások
használata változtatható feszültség forrásra van kapcsolva. 
Ebben az esetben viszont csak digitális feszültség szinteket lehet
használni, mivel az a rendszer nem alkalmaz egy DAC-ot.

Mindkét rendszer képes kiírni egy kijelzőre ami a teszteren található
a mérések eredményeit. Viszont a hasonló eszköz nemcsak a lábkiosztását,
hanem az áramköri rajzát is megjeleníti.

Viszont a hasonló eszköz nem képes változtatható feszültséggel vezérelni a
komponenst, így nem képes karakterisztika diagramok kirajzolására.

\section{További fejleszési lehetőségek}

A rendszernek lehetősége volna komplexebb alkatrészek
tesztelésére is, viszont a 3.3V-os feszültség határ
korlátoz sok esetben is. Például a MOSFET-ek tesztelése nagyobb
feszültségre lenne szükség, hogy biztosan azonosíthatóak legyenek.
Ezen kívül több féle dióda és tranzisztor fajták vannak, 
amelyekre komplexebb azonosítási algoritmus szükséges.

A rendszerrel lehetséges lenne állítható frekvenciával tekercsek mérésére
is és különböző típusú jelek generálása.

Módosítás az áramkörön, hogy a teszter socket biztosabb legyen, mivel
jelenleg egyes esetekben nem érintkezhet megfelelően ezért hibás az azonosítás.

Az áramkör módosítása olyan módon, hogy lehetséges legyen SMD alkatrészeket
is könnyedén tesztelni.

A mérési tartomány kiterjesztése a mérő ellenállások cseréjével.




