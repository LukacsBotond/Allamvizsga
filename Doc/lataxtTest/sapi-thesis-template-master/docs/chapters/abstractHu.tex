Napjainkban a mikrovezérlős rendszereken sok mindenben megtalálhatóak és manapság nagy számítási kapacitással rendelkeznek, általánosan alkalmazhatóak sokféle különböző alkalmazásban. Rengeteg funkciójuk van, az egyszerű LED villogtatástól kezdve komplex rendszerek automatizálásáig. Ezen kívül könnyű külső kiegészítő tartozékokat amelyekkel sokkal szélesebb körben használhatóak.

Ennek hatására a dolgozat célkitűzése egy olyan rendszer kialakítása, amely képes meghatározni egyszerű elektronikai komponenseket és azok megközelítő értékét és ezen kívül a lábkiosztását is amennyiben ez szükséges. Az elektronikában sok féle egyszerű komponenssel találkozhatunk, mint ellenállások, kondenzátorok, tranzisztorok. Viszont egy áramkör építésénél jó tudni, hogy az a komponens pontosan mi, ez legfőképpen igaz a különböző félvezetőkre. Sok esetben az azonosítója lekopott, vagy nem található adatlap így nehéz beazonosítani, hogy pontosan mi az a komponens. Erre szolgál az „elektronikai alkatrész teszter” amely automatikusan meghatározza, vagy kiírja, hogy hibás alkatrész ha nem ismeri fel vagy sérült az tesztelt alkatrész. A rendszer egy mikrovezérlőt és egy kijelzőt használ az komponens azonosítására és arról levő adatok kijelzésére a felhasználó felé. Az azonosítás teljesen automata, csupán csatlakoztatni kell az ismeretlen komponenst és egy gombot megnyomni.

A dolgozatban a mikrovezérlős alkalmazásokról és azok tervezéséről, alkatrészek felismeréséről és méréséről lesz szó.

\textbf{Kulcsszavak}: mikorvezérlő