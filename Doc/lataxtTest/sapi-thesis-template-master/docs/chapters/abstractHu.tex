Napjainkban a mikrovezérlős rendszereken sok mindenben megtalálhatóak és
nagy számítási kapacitással rendelkeznek, általánosan
alkalmazhatóak sokféle különböző alkalmazásban. Rengeteg funkciójuk van,
az egyszerű LED kapcsolgatástól kezdve komplex rendszerek automatizálásáig.
Ezen kívül könnyű külső kiegészítő tartozékokat amelyekkel sokkal
szélesebb körben használhatóak.

Ennek hatására a dolgozat célkitűzése egy olyan rendszer kialakítása,
amely képes meghatározni egyszerű elektronikai komponenseket és azok
megközelítő értékét és ezen kívül a lábkiosztását is. 
Az elektronikában sokféle egyszerű komponenssel találkozhatunk,
mint ellenállások, kondenzátorok, diódák, tranzisztorok. 
Viszont egy áramkör
építésénél jó tudni, hogy az a komponens pontosan mi, ez legfőképpen
igaz a különböző félvezetőkre. Sok esetben az azonosítója lekopott,
vagy nem található adatlap így nehéz beazonosítani, hogy pontosan mi
az a komponens. Erre szolgál az „elektronikai alkatrész teszter” amely
automatikusan meghatározza és kiírja az alkatrész értékeit, vagy 
hogy hibás/nem ismeri fel a tesztelt alkatrész. A rendszer egy mikrovezérlőt
és egy kijelzőt használ az komponens azonosítására és arról levő adatok
kijelzésére a felhasználó felé. Az azonosítás teljesen automata,
csupán csatlakoztatni kell az ismeretlen komponenst és egy kapcsolót
váltani, vagy automatikusan indulva tápfeszültségre csatlakozáskor.

A dolgozatban a mikrovezérlős alkalmazásokról és azok tervezéséről,
alkatrészek felismeréséről és méréséről lesz szó.

\textbf{Kulcsszavak}: mikrovezérlő, tranzisztor, azonosítás