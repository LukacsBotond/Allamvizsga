În zilile noastre, systeme de microcontrolere pot fi găsite în multe aplicațile
și avea o mare capacitate de calcul. În general pot fi utiliyate într-o
mare verietate de aplicații. Au o mulțime de funcții, de la simpla comutare 
a LED-urilor la automatizarea sistemelor complexe.
În plus pot utiliza accesorii externe cu care pot fi folosite în
mai multe aplicațile.

Ca urmare, scopul disertației este de a dezvolta un sistem care să poată 
determina componente electronice simple și valoarea lor aproximativă, 
precum și alocarea piciorului acestora.
Există multe componente simple în electronică, cum ar fi rezistoare, 
condensatoare, diode, tranzistoare.
Cu toate acestea, atunci când construiți un circuit, este bine să știți 
exact care este acea componentă, acest lucru este valabil mai ales pentru 
diferiți semiconductori.
În multe cazuri, ID-ul componente nu este visibil or fica techica
nu pot fi găsite.
Acesta este scopul „testerului de componente electronice” care determină și 
imprimă automat valorile componentei sau că componenta testată este defectă/nu 
este recunoscută.
Sistemul folosește un microcontroler și un ecran pentru a identifica 
componenta și a afișa datele acesteia către utilizator.
Identificarea este complet automată, tot ce trebuie să faceți este să 
conectați o componentă necunoscută și să comutați un comutator sau să 
porniți automat când este conectat la curent.

În disertației se va ocupa de aplicațiile microcontrolerelor și proiectarea acestora, 
recunoașterea și măsurarea componentelor.

\textbf{Cuvinte cheie}: microcontrolere, tranzistoare, Identificare
