A mikrovezérlős rendszerek manapság az életünk minden részében megtalálhatóak, kis méretük, 
alacsony áruk és meglehetősen nagy teljesítményükkel sok mindenre általánosan használhatóak. 
Ez nagyban csökkenti a tervezési költségeket, mivel nem kell egy specifikus logikai áramkört 
kialakítani minden egyes alkalmazási területre, csupán új program kódot kell feltölteni és 
használható egy teljesen más célra.

A mikrovezérlő könnyen összeköthetőek külső eszközökkel amellyel rengeteg mindent meg lehet 
valósítani és bővíteni a lehetőség és szabadon vezérelhetők a GPIO-n (General Purpose Input 
Output) keresztül sok mindent el lehet érni, az egyszerű LED kapcsolgatásától komplex jelek 
generálásáig. Általános esetben a mikrovezérlők csak a legfontosabb részeket tartalmazzák, 
mint az Analog Digital Converter amivel egy analóg jelet alakít egy digitális jellé amit a 
processzor fel tud majd dolgozni, ez legfőképpen azért van, mert nem mindenkinek van szüksége 
mindenre így akinek szüksége van az egyszerűen az külsőleg csatolja hozzá. 

Az összeállított rendszert szabadon lehet vezérelni így automatizálni lehet vele folyamatokat,
mivel egyszerű megismételhető teszteket végezni velük, miközben képesek valós időben mérni 
a rendszer viselkedését. Ennek a feladatnak is ez a lényege, az alkatrészek tesztelése egy 
gyors és automatizált módon. 


\section{Téma meghatározása}

A dolgozat célja egy olyan eszköz tervezése, amit bárki elektronikai ismeret nélkül is egyszerűen 
használni lehet. Sok esetben a feliratok az alkatrészeken nehezen látható, lekopott vagy 
egyszerűen nincs feltüntetve. Ilyen esetben sok segítséget tud nyújtani egy olyan eszköz ami 
gyorsan meg tudja határozni a komponenst és a lábkiosztását is amennyiben ez fontos.

Ez különösen nagy segítséget nyújt kezdőknek akik még kevésbé ismerik az alkatrészeket és az 
adatlapjai meg nagyok és komplexek számukra és a fő információk megjelenítése néhány sorban. 
Haladóknak is nagy segítség, mivel az ellenállást színkódjátról egyszerű meghatározni, viszont 
a teszterrel meg lehet határozni, hogy az alkatrész hibás-e, vagyis ha a tranzisztor kiégett 
akkor az is letesztelhető.

A teszternek 3 teszt terminálja van, ebbe kell az ismeretlen komponenst bekötni és képes 
egyszerű elektronikai komponensek (ellenállás, dióda, tranzisztorok, stb.) automatikus 
felismerését és az adatainak meghatározására. Viszont nem képes bonyolultabb áramkörök 
azonosítására aminek összesen több mint 3 lába van.

Megvalósítás során a költégek csökkentése a cél, miközben a pontosság nem csökken nagyban. 
Két verzió is összeállítható, az egyik egyszerű ellenállásokkal és a második egy DAC (Digital 
Analog Converter) segítségével. Mindkettő alkalmas a komponensek meghatározására, viszont az 
ellenállásos verzió nem alkalmas karakterisztika diagram kirajzolására, viszont sokkal olcsóbb, 
mivel nem használ egy külső DAC-ot.

Mérés eredménye kikerül egy kis kijelzőre és grafikus felületen is megtekinthető amennyiben 
egy számítógéphez van csatolva. Viszont a 2 közül legalább az egyikre szükség van, különben a 
mérés eredménye nem lesz látható. A kijelzőt nem kötelező alkalmazni, viszont annélkül csak 
egy laptop/számítógéphez kapcsolva lehet használni.

Ehhez szükséges egy processzor, viszont manapság a mikrovezérlők nagy számítási kapacitással 
rendelkeznek meglehetősen alacsony áron és néhány ellemállásból és vezetékből otthon is 
összeállítható.

Karakterisztika diagramm kirajzolása is fontos, viszont ez leginkább a tranzisztoroknál 
fontos.
